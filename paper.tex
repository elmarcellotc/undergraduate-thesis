\documentclass[12pt]{article}
\usepackage{mathptmx}
\usepackage[left=1in, right=1in, top=1in, bottom=1in]{geometry}
\usepackage{titling}
\pretitle{\begin{center}\Huge\bfseries}
    \posttitle{\par\end{center}\vskip 0.5em}
    \preauthor{\begin{center}\Large\ttfamily}
    \postauthor{\end{center}}
    \predate{\par\large\centering}
    \postdate{\par}

\title{UNIVERSITY SAN FRANCISCO OF QUITO}
\begin{document}
\maketitle
\begin{abstract}

     For businessmen, public policy makers, and investors, it
    is required to know the latest innovations and which
    kind of them must be considered, and how much. There exists some attempts
    to predict which will be the future innovation at one
    specific industry. The Economic Complexity has shown itself
    as a good tool to solve this problem based on reduction of
    dimensionality techniques. On this working paper, 
    I characterized the Ecuadorian productive and investigative
    networks replicating the Economic Complexity
    methodology. Having the industrial networks, my contribution 
    is to generated links between the two Networks. Thus, it is 
    possible to model a system of recommendation of future 
    industries in which an agent can decide how to assign 
    resources with better information from a specific
    sector the growth.
\end{abstract}
\end{document}