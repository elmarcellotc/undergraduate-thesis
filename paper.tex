\documentclass[12pt]{article}
\usepackage{mathptmx}
\usepackage[left=1in, right=1in, top=1in, bottom=1in]{geometry}
\usepackage{titling}
\usepackage{setspace}

\pretitle{\begin{center}\Huge\bfseries}
    \posttitle{\par\end{center}\vskip 0.5em}
    \preauthor{\begin{center}\Large\ttfamily}
    \postauthor{\end{center}}
    \predate{\par\large\centering}
    \postdate{\par}

\title{UNIVERSITY SAN FRANCISCO OF QUITO}
\doublespacing
\begin{document}
\maketitle
\begin{abstract}

    For businessmen, public policy makers, and investors, it
is required to know the latest innovations and which
kind of them must be considered, and how much. There exists some attempts
to predict which will be the future innovation at one
specific industry. The Economic Complexity has shown itself
as a good tool to solve this problem based on reduction of
dimensionality techniques. On this working paper, 
I characterized the Ecuadorian productive and investigative
networks replicating the Economic Complexity
methodology. Having the industrial networks, my contribution 
is to generated links between the two Networks. Thus, it is 
possible to model a system of recommendation of future 
industries in which an agent can decide how to assign 
resources with better information from a specific
sector the growth.
    
\end{abstract}

\section{INTRODUCTION}

\section{METHODOLOGY}

\subsection{THE ECONOMIC COMPLEXITY}

The traditional economics is center on the input output
relationship over aggregate data. Some examples are
the GDP, the labor's rate, direct investment, and others.
The main difference between the traditional
Economics and the Economic Complexity is on the use of
disaggregated data (Hidalgo, 2021). This approach is needed
to provide a good understanding of how the relationship between
the economic agents affect the aggregated data. Whereas in the
traditional approach the questions are centered on the 
\textit{Ceteris Paribus} analysis, the complexity approach 
is focused on how much distance exist on the inputs and 
which is the marginal effect of that distance on the
economical output.

There exist a dilemma between prioritize to diversify or
to specialize the production. Futhermore, there exist
products that are more demanded but
less actors are producing them. Exist evidence that the
reflection methodology provide information about the
capabilities available in a Country searching on the required
capabilities for a good production (Hidalgo \& Hausmann, 2009).
Also, the Revealed Comparative Advantage (RCA) has been used
to know if some relationship between a producer and a product,
or a consumer, is significant.

 



\end{document}